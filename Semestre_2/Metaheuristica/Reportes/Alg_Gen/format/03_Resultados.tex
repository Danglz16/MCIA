\section{Resultados}
% Se muestran e interpretan los resultados como se obtuvieron

En la figura~\ref{PoblacionInicial} podemos ver la población inicial generada aleatoriamente, donde el individuo con mayor aptitud es 0.87 
mientras se desarrolla el AG podemos observar como la aptitud promedio mejora generación tras generación, sin embargo, no siempre el mejor 
individuo de una generación es mejor que el de la generación anterior, de hecho en este caso como podemos ver más adelante en la figura~\ref{Gen_final},
el mejor individuo de la generacio 0 es decir nuestra población inicial, se mantiene como el mejor individuo durante todas las generaciones.

Esto puede ser un indicativo de que el AG no está explorando lo suficiente el espacio de soluciones, por otro lado esto seria algo esperado ya que
no se implementó ninguna técnica de mutacion, por lo que el AG se basa únicamente en la selección y el cruce para generar nuevas soluciones.
En la figura~\ref{Grafica} podemos observar como la aptitud promedio de la población mejora generación tras generación, pero el maximo se mantiene constante
debido a que el mejor individuo de la población inicial se mantiene como el mejor individuo durante todas las generaciones.

Podemos ver que el promedio empieza a converger hacia el maximo, lo que indica que la población está convergiendo hacia una solución óptima, sin embargo, no
podemos asegurar que esta solución es la óptima ya que despues de unas generaciones se estanca.

Por otro lado al implementar el segundo metodo de cruce (2 puntos) podemos observar en la figura~\ref{Pob_final_2ptos} un comportamiento similar donde el mejor
individuo de la población inicial se mantiene como el mejor individuo durante todas las generaciones, sin embargo, es importante destacar que el promedio de la población
sube rapidamente en las primeras generaciones y depues se estanca, debido a que la población ya no mejora.