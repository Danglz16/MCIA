\section{Discusión}
% Interpretacion propia de los datos

A lo largo del desarrollo del proyecto podemos ver que la implementacion correcta de cada etapa es primordial para el correcto funcionamiento
del AG esto toma especial relevancia al momento de mencionar que no se implemento la mutacion ya que este paso nos permitiria tener un mayor
grado de exploracion del espacio de soluciones, lo cual podria llevarnos a encontrar mejores soluciones en menos generaciones y de esta manera
evitar el estancamiento, ademas esto podria hacer que nuestro super individuo de la generacion 0 se mantenga siempre como el mejore.

Otro punto a destacar es como el cambio en el metodo de cruce afecta el rendimiento del algoritmo, en este caso el cruce por un punto
mostro ser mas efectivo que el cruce con dos puntos, esto puede deberse a que al tener un solo punto de corte se mantiene una mayor 
parte de la informacion genetica de los padres, lo que permite que las soluciones generadas sean mas cercanas a las mejores soluciones 
encontradas hasta el momento. Aunque por otro lado el usar dos puntos hace que la aptitud crezca mas rapidamente, pero esto puede llevar a que
el algoritmo se estanque en un optimo local con la misma velocidad como fue el caso.