\section{Conclusión}
% Que puedo decir con base en lo que obtuve

Los procesos EVOP al ser un modelo flexible que permite el ajuste a lo largo de la ejecución nos presenta un precursor de los algoritmos genéticos, esto debido a que aunque se plantea desde un inicio a lo largo de la ejecución puede evolucionar, tal como su nombre lo indica y sin embargo llegar al resultado esperado haciendo pequeños ajustes.
Este proceso ofrece una gran herramienta a la hora de tratar de mejorar un proceso ya establecido con una meta clara y sin un análisis complejo previo por lo cual, es de gran ayuda principalmente en pequeños y medianos proyectos, ya que para empresas más grandes podría causar incertidumbre y en lineas grandes podría causar muchas perdidas.
Para este caso en especifico se tomo como punto de partida una linea de producción de pan esto nos permite aterrizarlo de manera más clara al mundo real.
Los valores de las variables fueron propuestos pensando en valores que tengan un impacto significativo en el producto final, de la misma manera es importante tener en cuenta que cada cambio asi como estas variaciones pueden ser modificadas hasta llegar a un punto optimo.
Es decir no tiene que permanecer inmutables, llegado el momento los cambios pueden ser menores o mayores segun sea el caso.