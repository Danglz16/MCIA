\section{Introducción}

La operación evolutiva (EvOp por sus siglas en ingles) es una técnica de mejora continua desarrollada en los años 50 por George E.P. Box, su objetivo es optimizar procesos sin detener la operación y se basa en principios evolutivos, comúnmente utilizada en procesos industriales, químicos o de calidad mediante experimentación y ajuste continuo\cite{barnett1960introduction}.

Durante el desarrollo de esta metodología debemos identificar nuestras variables o parámetros para poder encontrar soluciones optimas. El objetivo de un evop puede variar sin embargo el enfoque es el mismo, optimizar procesos en tiempo real sin necesidad de un modelo previo adaptándose a condiciones cambiantes y con poco o nulo control del entorno\cite{box1957evolutionary}.

\subsection*{Puntos clave para la implementación de un EVOP}

\begin{itemize}
    \item \textbf{Selección del Sistema y Variables a Optimizar:} Identificar claramente el proceso o sistema a optimizar y las variables controlables críticas que afectan el desempeño.
    
    \item \textbf{Diseño Experimental Inicial:} Establecer condiciones iniciales, definir rangos de variación para las variables y determinar cómo se va a medir el rendimiento o la función objetivo.
    
    \item \textbf{Iteración y Evaluación Continua:} Cambiar sistemáticamente las variables según reglas evolutivas (mutación, selección) y evaluar el desempeño para determinar cambios beneficiosos.
    
    \item \textbf{Balance entre Exploración y Explotación:} Equilibrar la búsqueda de nuevas soluciones (exploración) con la optimización de las ya encontradas (explotación) para evitar estancarse en soluciones subóptimas.
    
    \item \textbf{Adaptabilidad y Flexibilidad:} Permitir que el sistema evolucione ante cambios en el entorno o en el propio proceso, manteniendo la capacidad de respuesta y ajuste dinámico.
    
    \item \textbf{Recopilación y Análisis de Datos:} Monitorear continuamente los resultados para apoyar la toma de decisiones y la siguiente ronda de ajustes.
    
    \item \textbf{Consideraciones Prácticas:} Evaluar costos, tiempo de experimentación, recursos disponibles y riesgos asociados con los cambios durante la operación.
\end{itemize}
