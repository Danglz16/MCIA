\section{Desarrollo}
% Metodologia / Desarrollo
En nuestro caso se decidió optimizar el proceso de horneado en una panadería mientras la operación continúa normalmente. Se evalúan pequeños cambios en variables de control para reducir consumo energético y mejorar calidad sin afectar la producción.

\subsection{Objetivo}

\subsubsection*{Objetivo principal}
Reducir el consumo específico de energía por bandeja en al menos 5\% durante el horneado, sin perder la calidad del producto ni el rendimiento de la línea.

\subsubsection*{Criterios de éxito}
\begin{itemize}
  \item \textbf{Energía:} reducción media del consumo específico de energía $E_{\mathrm{esp}}$ $\geq 5\% $ respecto a la línea base (prueba \textit{t} bilateral, $p<0{.}05$).
  \item \textbf{Calidad:} 
    \begin{itemize}
      \item Humedad interna $M$ dentro de $\pm 0{.}5$ puntos porcentuales del valor objetivo.
      \item Color de corteza (escala 1--5): dentro de tolerancia definida (p.\,ej., $\pm 1$ punto o $\pm 2$ unidades $L^*$) puede ser medido con un higrómetro.
      \item Defectos $D$ (\,\%\,)  no mayores a la línea base (prueba de proporciones, $p\ge 0{.}05$ para no-inferioridad).
    \end{itemize}
  \item \textbf{Productividad:} no reducir la salida de bandejas/hora en más de $1\%$ ni aumentar el tiempo de ciclo en más de 1 minuto.
\end{itemize}

\subsubsection*{Métricas y definiciones}
\[
E_{\mathrm{esp}}=\frac{\text{Energía consumida (kWh)}}{\text{Nº de bandejas horneadas}},\quad
D=\frac{\text{Piezas defectuosas}}{\text{Piezas totales}}\times 100\%.
\]
El volumen específico (opcional) se estima como $V_{\mathrm{esp}}=\frac{\text{Volumen del pan}}{\text{Peso}}$ (mL/g). 

\subsubsection*{Alcance }
El EVOP se ejecutará durante \textbf{4 semanas}: 1 de preparación y 3 para ciclos en operación normal, con cambios pequeños y controlados.

\subsubsection*{Restricciones}
Mantener los parámetros dentro de límites de seguridad del horno y especificaciones del producto; todos los cambios serán reversibles y dentro de tolerancias.

\subsection{Variables del EVOP}
\subsubsection*{Factores de control (ajustables)}
\begin{itemize}
  \item Temperatura de horneado $T$ (\si{\celsius}):  $\Delta T={\SI{10}{\celsius}}$ alrededor de $T_0={\SI{155}{\celsius}}$.
  \item Tiempo de horneado $H$ (\si{\minute}): $\Delta H={\SI{10}{\minute}}$ alrededor de $H_0={\SI{55}{\minute}}$.
\end{itemize}

\subsubsection*{Variables respuesta (medibles en producción)}
\begin{itemize}[left=0pt]
  \item Consumo específico de energía $E$ (\si{\kilo\watt\hour} por bandeja).
  \item Humedad interna $M$ (\si{\percent}).
  \item Color de corteza (escala visual 1--5).
  \item Porcentaje de defectos $D$ (\si{\percent}).
\end{itemize}

\subsection{Diseño y Ejecución}
Se proponen ajustes en nuestras variables para aplicar durante la operación normal. Cada ciclo cubre las cuatro combinaciones y se repite $C={4}$ veces.

\begin{table}[H]\centering
\caption{Matriz EVOP (codificada y real).}
\begin{tabular}{cccccc}
\toprule
\multirow{2}{*}{Corrida} & \multicolumn{2}{c}{Factores codificados} & \multicolumn{2}{c}{Ajustes reales} \\
\cmidrule(r){2-3}\cmidrule(l){4-5}
 & $x_T$ & $x_H$ & $T$ (\si{\celsius}) & $H$ (\si{\minute}) & \\
\midrule
1 & $+1$ & $+1$ & $T_0+\Delta T$ & $H_0+\Delta H$ \\
2 & $+1$ & $-1$ & $T_0+\Delta T$ & $H_0-\Delta H$ \\
3 & $-1$ & $+1$ & $T_0-\Delta T$ & $H_0+\Delta H$ \\
4 & $-1$ & $-1$ & $T_0-\Delta T$ & $H_0-\Delta H$ \\
\bottomrule
\end{tabular}
\end{table}

También se considera no aplicar la corrida siempre en el mismo orden, ya que esto puede generar sesgos de tiempo o temperatura del ambiente. Por lo cual debe ejecutarse en orden aleatorio.

Con cada corrida se registraran los resultados en la siguiente tabla:
\begin{table}[H]\centering
\caption{Formato de registro por corrida (un extracto).}
\begin{tabular}{cccccccc}
\toprule
Ciclo & Corrida & $T$ & $H$ & $E$ & $M$ & Color & $D$\\
 &  & (\si{\celsius}) & (\si{\minute}) & (\si{\kilo\watt\hour/bandeja}) & (\si{\percent}) & (1--5) & (\si{\percent}) \\
\midrule
1 & 1 &  &  &  &  &  &  \\
1 & 2 &  &  &  &  &  &  \\
1 & 3 &  &  &  &  &  &  \\
1 & 4 &  &  &  &  &  &  \\
\midrule
\multicolumn{8}{l}{\footnotesize Observaciones: {...}}\\
\bottomrule
\end{tabular}
\end{table}

\subsection{Análisis de Resultados}
\subsubsection*{Modelo lineal con interacción}
Para una respuesta $Y$ (p.ej., $E$ o $D$), usar:
\begin{equation}
Y = \beta_0 + \beta_T x_T + \beta_H x_H + \beta_{TH} x_T x_H + \varepsilon,
\end{equation}
donde $x_T,x_H\in\{-1,+1\}$. Un efecto $\beta$ significativamente distinto de cero indica dirección de mejora.

\subsubsection*{Contrastes de efectos (cálculo manual rápido)}
Promedios en alto/bajo:
\[
\hat{\beta}_T=\frac{\bar{Y}_{x_T=+1}-\bar{Y}_{x_T=-1}}{2},\quad
\hat{\beta}_H=\frac{\bar{Y}_{x_H=+1}-\bar{Y}_{x_H=-1}}{2},\quad
\hat{\beta}_{TH}=\frac{\bar{Y}_{x_Tx_H=+1}-\bar{Y}_{x_Tx_H=-1}}{2}.
\]
Repetir por ciclo y promediar sobre $C$ ciclos. Visualizar con gráficos de efectos e intervalos de confianza.

\subsubsection*{Ejemplo}
Se evalúa el consumo específico de energía $Y$ (kWh/bandeja) con dos factores codificados:
$x_T\in\{-1,+1\}$ (temperatura: $\pm 10\,^{\circ}\mathrm{C}$) y
$x_H\in\{-1,+1\}$ (tiempo: $\pm 10$ min).

\begin{table}[H]\centering
\caption{Corridas del diseño y respuesta observada $Y$.}
\begin{tabular}{cccc}
\toprule
Corrida & $x_T$ & $x_H$ & $Y$ (kWh/bandeja) \\
\midrule
1 & +1 & +1 & 2.60 \\
2 & +1 & -1 & 2.45 \\
3 & -1 & +1 & 2.55 \\
4 & -1 & -1 & 2.40 \\
\bottomrule
\end{tabular}
\end{table}

\subsubsection*{Modelo}
El modelo lineal con interacción es:
\[
Y = \beta_0 + \beta_T x_T + \beta_H x_H + \beta_{TH} x_T x_H + \varepsilon.
\]
\subsubsection*{Estimación por contrastes (codificado)}
Promedio general:
\[
\hat{\beta}_0 = \bar{Y} = \frac{2.60+2.45+2.55+2.40}{4} = 2.50.
\]
Efectos principales:
\[
\hat{\beta}_T =
\frac{\bar{Y}\,\big|\,x_T=+1 - \bar{Y}\,\big|\,x_T=-1}{2}
=
\frac{\frac{2.60+2.45}{2} - \frac{2.55+2.40}{2}}{2}
=
\frac{2.525 - 2.475}{2} = 0.025,
\]
\[
\hat{\beta}_H =
\frac{\bar{Y}\,\big|\,x_H=+1 - \bar{Y}\,\big|\,x_H=-1}{2}
=
\frac{\frac{2.60+2.55}{2} - \frac{2.45+2.40}{2}}{2}
=
\frac{2.575 - 2.425}{2} = 0.075.
\]
Interacción:
\[
\hat{\beta}_{TH} =
\frac{\bar{Y}\,\big|\,x_Tx_H=+1 - \bar{Y}\,\big|\,x_Tx_H=-1}{2}
=
\frac{\frac{2.60+2.40}{2} - \frac{2.45+2.55}{2}}{2}
=
\frac{2.50 - 2.50}{2} = 0.
\]

\subsubsection*{Modelo ajustado (codificado)}
\[
\hat{Y} = 2.50 + 0.025\,x_T + 0.075\,x_H + 0\cdot(x_Tx_H).
\]

\subsubsection*{Interpretación (codificado)}
\begin{itemize}
  \item $\hat{\beta}_0=2.50$: consumo promedio en los set-points centrales ($T_0, H_0$).
  \item $\hat{\beta}_T=0.025$: mover la temperatura de $T_0$ a $T_0{+}10^{\circ}\mathrm{C}$ (cambio de $x_T=-1$ a $+1$) incrementa $Y$ en $0.025$ kWh/bandeja.
  \item $\hat{\beta}_H=0.075$: mover el tiempo de $H_0$ a $H_0{+}10$ min (cambio de $x_H=-1$ a $+1$) incrementa $Y$ en $0.075$ kWh/bandeja.
  \item $\hat{\beta}_{TH}=0$: no se detecta interacción en este rango.
\end{itemize}

\subsubsection*{Efectos en unidades naturales}
Como $x_T=\pm 1$ representa $\pm 10^{\circ}\mathrm{C}$ y $x_H=\pm 1$ representa $\pm 10$ min:
\[
\text{Efecto por }^{\circ}\mathrm{C}\ \approx \frac{\hat{\beta}_T}{10} = \frac{0.025}{10} = 0.0025\ \text{kWh/bandeja/}^{\circ}\mathrm{C},
\]
\[
\text{Efecto por minuto}\ \approx \frac{\hat{\beta}_H}{10} = \frac{0.075}{10} = 0.0075\ \text{kWh/bandeja/min}.
\]
\textbf{Implicación}: una disminución de $1$ min reduce en promedio $0.0075$ kWh/bandeja; una disminución de $1^{\circ}\mathrm{C}$ reduce en promedio $0.0025$ kWh/bandeja (siempre que la calidad permanezca dentro de tolerancias).

\subsection{Criterios de decisión}

\subsubsection*{Reglas de decisión (energía)}
\begin{enumerate}[label=\textbf{D\arabic*.}, left=0pt]
  \item \textbf{Dirección de mejora:} Si $\hat{\beta}_H>0$ (aumentar tiempo eleva $E_{\mathrm{esp}}$), entonces moverse hacia $x_H=-1$ (tiempos menores). Si $\hat{\beta}_T>0$, moverse hacia $x_T=-1$ (temperaturas menores). Priorizar el factor con mayor $|\hat{\beta}|$.
  \item \textbf{Interacción:} Si $|\hat{\beta}_{TH}|$ es significativo, la dirección óptima depende de la combinación $T$–$H$; en este caso, realizar un \textit{follow-up} focal con pasos más finos alrededor del mejor cuadrante.
  \item \textbf{Meta cuantitativa:} Aceptar el nuevo set-point si se demuestra una reducción de al menos \textbf{{$5\%$}} en $E_{\mathrm{esp}}$ respecto a la línea base, con $p<0{.}05$ o con intervalo de confianza de la diferencia completamente por debajo de 0.
\end{enumerate}

\subsubsection*{Salvaguardas de calidad (no-inferioridad)}
Para cada métrica de calidad:
\begin{itemize}[left=0pt]
  \item \textbf{Humedad $M$:} Mantener $\Delta M$ dentro de \textbf{$\pm{0{.}5}$ puntos porcentuales} respecto a objetivo. Prueba de no-inferioridad con margen $\delta_M={0{.}5}$ p.p.
  \item \textbf{Color:} 
    \begin{itemize}
      \item Escala 1--5: diferencia media dentro de $\pm{1}$ punto.
    \end{itemize}
  \item \textbf{Defectos $D$:} No mayor que la línea base (prueba de proporciones; margen de no-inferioridad $\delta_D={0{.}5}$ p.p.).
\end{itemize}

\subsubsection*{Regla de aceptación integral}
\begin{quote}
\textbf{Aceptar} el nuevo set-point si:(i) Disminuye el costo energético, y \\(ii) $M$, Color y $D$ no exceden en disminución dentro de los márgenes definidos.
\end{quote}

\subsection{Implementación y verificación}

\subsubsection*{Plan de implementación}
\begin{enumerate}[left=0pt]
  \item \textbf{Capacitación y checklist} (Semana 1): calidad y mantenimiento; verificación de sensores y límites de seguridad; checklist operativo completado.
  \item \textbf{Aplicación de set-points} (Semanas 2--3): implementar el plan cíclico EVOP (diseño $2^2$ a baja amplitud), con orden aleatorizado en cada ciclo y separación entre corridas de {\SI{1}{\minute}} para estabilización térmica.
  \item \textbf{Registro de datos}: capturar $E_{\mathrm{esp}}$, $M$, Color, $D$, temperatura ambiental y observaciones de masa.
  \item \textbf{Análisis intermedio}: al cierre de cada ciclo, revisar efectos y cumplimiento de salvaguardas; detener o ajustar si se exceden límites.
\end{enumerate}

\subsubsection*{Riesgos y planes de reversión}
\begin{itemize}[left=0pt]
  \item \textbf{Sobre/infra-cocción}: monitoreo de Color y $M$ por lote; si se excede tolerancia, revertir a $(T_0,H_0)$.
  \item \textbf{Variación de materia prima}: registrar humedad de masa; si cambia $>{1{.}0}$ p.p., pausar ajustes y recalibrar objetivo.
  \item \textbf{Inestabilidad térmica}: incremento de dispersión en $E_{\mathrm{esp}}$; verificar quemadores/ventiladores y sellos del horno antes de continuar.
\end{itemize}

